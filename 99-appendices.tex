\chapter{DCPU-16 Specification}
\VerbatimInput[fontsize=\footnotesize]{dcpu16.txt}
\chapter{Simplified J5 Instruction Sheet}
Transcribed from 2013 DACS Exam paper
\subsubsection{Arithmetic \& Logical}
\begingroup
\ttfamily
\begin{tabularx}{0.95\textwidth}{l X}
  ADD & Add Top \& Next on stack together   (3,2) -> (5) \\
  SUB & Subtract Top from Next on stack   (2,8) -> (6) \\
  INC & Increment Top of stack \\
  DEC & Decrement Top of stack \\
  AND/OR/NOT/XOR & Combine Top \& Next in in logical operation \\
  SHR & Shift Top of stack 1 bit to right \\
  SHL & Shift Top of stack 1 bit to left \\
\end{tabularx}
\endgroup

\subsubsection{Comparative}
\begingroup
\ttfamily
\begin{tabularx}{0.95\textwidth}{l X}
  TGT & Test Top greater than Next, set Zero flag to true or false* \\
  TLT & Test Top less than Next, set Zero flag to true or false* \\
  TEQ & Test Top equal to Next, set Zero flag to true or false* \\
  TSZ & Test Top set to zero, set Zero flag to true or false* \\
\end{tabularx}
\endgroup

\hfill*Note these do not destroy Top and Next when they are operated.

\subsubsection{Other instructions}
\begingroup
\ttfamily
\begin{tabularx}{0.95\textwidth}{l >{\raggedright\arraybackslash}X}
  SSET n & Place an operand n on the stack \\
  SET nn & Place an operand nn on the stack \\
  LOAD & Take address from Top of stack, access location, put result on stack \\
  STORE & Take Top as address, and Next as data, then write data to the address
  \\
  BRANCH nn & Unconditional relative branch of +/- nn bits \\
  BRZERO nn & Conditional branch (taken if Zero Flag set) \\
  IBRANCH & Indirect branch PC <= M[TOS] \\
  CALL nn & Call to address nn \\
  RETURN & Return from call \\
  STOP & Causes Execution to halt \\
\end{tabularx}
\endgroup

\subsubsection{Stack manipulations}
\begingroup
\ttfamily
\begin{tabularx}{\textwidth}{l X r}
  DROP & Drop an item from the Top of the stack & \\
  DUP & Duplicate Top of stack & (X,Y) -> (X,X,Y) \\
  SWAP & Swap Top and Next & (X,Y) -> (Y,X) \\
  RSD3 & Rotate stack down three & (X,Y,Z) -> (Z,X,Y) \\
  RSU3 & Rotate stack up three & (X,Y,Z) -> (Y,Z,X) \\
  TUCK2 & Tuck copy of Top under 2nd item & (X,Y,Z) -> (X,Y,X,Z) \\
  TUCK3 & Tuck copy of Top under 3rd item & (X,Y,Z) -> (X,Y,Z,X) \\
  COPY3 & Copy 3rd stack item to Top of stack & \\
  PUSH rr & Push register (F,LBR,GBR,VBA) onto stack & \\
  POP rr & Pop register (F,LBR,GBR,VBA) from stack & \\
\end{tabularx}
\endgroup

\subsubsection{Registers}
The J5 has the usual stack registers, Top, Next, 3rd, etc, and also has Local
Base Register (\textbf{LBR}), Global Base Register (\textbf{GBR}), Vector Base
Address (\textbf{VBA}), and an F register containing the flags shown below.

\vspace{10mm}

\noindent\begin{minipage}[b]{0.5\linewidth}
  \begin{tikzpicture}
    \tikzstyle{every path}=[thick]
    \edef\turingtapesize{0.7cm}
    \tikzstyle{tmtape}=[draw,minimum size=\turingtapesize]

    \begin{scope}[start chain=1 going right,node distance=0mm]
      \node[on chain=1,tmtape,label=below:MSB] {I};
      \node[on chain=1,tmtape]                 {M};
      \node[on chain=1,tmtape]                 {};
      \node[on chain=1,tmtape]                 {};
      \node[on chain=1,tmtape]                 {};
      \node[on chain=1,tmtape]                 {};
      \node[on chain=1,tmtape]                 {Z};
      \node[on chain=1,tmtape,label=below:LSB] {C};
    \end{scope}
  \end{tikzpicture}
\end{minipage}%
\noindent\begin{minipage}[b]{0.5\linewidth}
  \textbf{C}-Carry Flag\@ \textbf{Z}-Zero Flag\\
  \textbf{I}-Interrupt (1=Enable, 0=Disable)\\
  \textbf{M}-Interrupt Mode (1=Vector, 0=Opcode)
\end{minipage}
