\chapter{DCPU-16 Specification}\label{ch:dcpuspec}
\VerbatimInput[fontsize=\footnotesize]{dcpu16.txt}

\chapter{Simplified J5 Instruction Sheet}\label{ch:j5spec}
Transcribed from 2013 DACS Exam paper
\subsubsection{Arithmetic \& Logical}
\begingroup
\ttfamily
\begin{tabularx}{0.95\textwidth}{l X}
  ADD & Add Top \& Next on stack together   (3,2) -> (5) \\
  SUB & Subtract Top from Next on stack   (2,8) -> (6) \\
  INC & Increment Top of stack \\
  DEC & Decrement Top of stack \\
  AND/OR/NOT/XOR & Combine Top \& Next in in logical operation \\
  SHR & Shift Top of stack 1 bit to right \\
  SHL & Shift Top of stack 1 bit to left \\
\end{tabularx}
\endgroup

\subsubsection{Comparative}
\begingroup
\ttfamily
\begin{tabularx}{0.95\textwidth}{l X}
  TGT & Test Top greater than Next, set Zero flag to true or false* \\
  TLT & Test Top less than Next, set Zero flag to true or false* \\
  TEQ & Test Top equal to Next, set Zero flag to true or false* \\
  TSZ & Test Top set to zero, set Zero flag to true or false* \\
\end{tabularx}
\endgroup

\hfill*Note these do not destroy Top and Next when they are operated.

\subsubsection{Other instructions}
\begingroup
\ttfamily
\begin{tabularx}{0.95\textwidth}{l >{\raggedright\arraybackslash}X}
  SSET n & Place an operand n on the stack \\
  SET nn & Place an operand nn on the stack \\
  LOAD & Take address from Top of stack, access location, put result on stack \\
  STORE & Take Top as address, and Next as data, then write data to the address
  \\
  BRANCH nn & Unconditional relative branch of +/- nn bits \\
  BRZERO nn & Conditional branch (taken if Zero Flag set) \\
  IBRANCH & Indirect branch PC <= M[TOS] \\
  CALL nn & Call to address nn \\
  RETURN & Return from call \\
  STOP & Causes Execution to halt \\
\end{tabularx}
\endgroup

\subsubsection{Stack manipulations}
\begingroup
\ttfamily
\begin{tabularx}{\textwidth}{l >{\raggedright\arraybackslash}X r}
  DROP    & Drop an item from the Top of the stack   & \\
  DUP     & Duplicate Top of stack                   & (X,Y) -> (X,X,Y)     \\
  SWAP    & Swap Top and Next                        & (X,Y) -> (Y,X)       \\
  RSD3    & Rotate stack down three                  & (X,Y,Z) -> (Z,X,Y)   \\
  RSU3    & Rotate stack up three                    & (X,Y,Z) -> (Y,Z,X)   \\
  TUCK2   & Tuck copy of Top under 2nd item          & (X,Y,Z) -> (X,Y,X,Z) \\
  TUCK3   & Tuck copy of Top under 3rd item          & (X,Y,Z) -> (X,Y,Z,X) \\
  COPY3   & Copy 3rd stack item to Top of stack      & \\
  PUSH rr & Push register (F,LBR,GBR,VBA) onto stack & \\
  POP rr  & Pop register (F,LBR,GBR,VBA) from stack  & \\
\end{tabularx}
\endgroup

\subsubsection{Registers}
The J5 has the usual stack registers, Top, Next, 3rd, etc, and also has Local
Base Register (\textbf{LBR}), Global Base Register (\textbf{GBR}), Vector Base
Address (\textbf{VBA}), and an F register containing the flags shown below.

\vspace{10mm}

\noindent\begin{minipage}[b]{0.5\linewidth}
  \begin{tikzpicture}
    \tikzstyle{every path}=[thick]
    \edef\turingtapesize{0.7cm}
    \tikzstyle{tmtape}=[draw,minimum size=\turingtapesize]

    \begin{scope}[start chain=1 going right,node distance=0mm]
      \node[on chain=1,tmtape,label=below:MSB] {I};
      \node[on chain=1,tmtape]                 {M};
      \node[on chain=1,tmtape]                 {};
      \node[on chain=1,tmtape]                 {};
      \node[on chain=1,tmtape]                 {};
      \node[on chain=1,tmtape]                 {};
      \node[on chain=1,tmtape]                 {Z};
      \node[on chain=1,tmtape,label=below:LSB] {C};
    \end{scope}
  \end{tikzpicture}
\end{minipage}%
\noindent\begin{minipage}[b]{0.5\linewidth}
  \textbf{C}-Carry Flag\@ \textbf{Z}-Zero Flag\\
  \textbf{I}-Interrupt (1=Enable, 0=Disable)\\
  \textbf{M}-Interrupt Mode (1=Vector, 0=Opcode)
\end{minipage}

\chapter{Benchmark programs}\label{ch:benchmarkprograms}
\begin{lstlisting}[caption={bsort.reg}]
; initial "input"
SET [0x3000], 6221
SET [0x3001], 16181
SET [0x3002], 29598
SET [0x3003], 10115
SET [0x3004], 18953
SET [0x3005], 44337
SET [0x3006], 12629
SET [0x3007], 7527
SET [0x3008], 56169
SET [0x3009], 44334
SET [0x300a], 2290
SET [0x300b], 38260
SET [0x300c], 62627
SET [0x300d], 9740
SET [0x300e], 40652
SET [0x300f], 58444
SET [0x3010], 21421
SET [0x3011], 45306
SET [0x3012], 38396
SET [0x3013], 25410
SET [0x3014], 31580
SET [0x3015], 17208
SET [0x3016], 27939
SET [0x3017], 59297
SET [0x3018], 55094
SET [0x3019], 45095
SET [0x301a], 55388
SET [0x301b], 30943
SET [0x301c], 33757
SET [0x301d], 48729
SET [0x301e], 42
SET [0x301f], 29832

SET [0x2fff], 32 ; length

; sort
SET J, [0x2fff]
:LOOP SUB J, 1
	SET I, 0
:LOOP2 SET A, [0x3000+I]
	SET B, [0x3000+J]
	IFG A, B ; compare
		SET PC, SWAP
	SET PC, POST
:SWAP SET C, [0x3000+I]
	SET [0x3000+I], [0x3000+J]
	SET [0x3000+J], C
:POST ADD I, 1
	IFG J, I
		SET PC, LOOP2
	IFN J, 0
		SET PC, LOOP

; output result
SET I, 0
:RESL OUT [0x3000+I]
	ADD I, 1
	IFN I, [0x2fff]
		SET PC, RESL
\end{lstlisting}

\begin{lstlisting}[caption={fib20.reg}]
SET [0x2000], 20 ; limit

SET A, 0 ; F(0)
SET B, 1 ; F(1)

OUT A
OUT B

SET I, 1
:LOOP SET C, A
	SET A, B
	ADD B, C
	OUT B
	ADD I, 1
	IFN I, [0x2000]
		SET PC, LOOP
\end{lstlisting}

\noindent\begin{minipage}{\linewidth}
\begin{lstlisting}[caption={primes.reg}]
SET I, 100 ; limit
SET A, 2
:LOOP IFE [A+3000], 0
	SET PC, SCAN
:SCANNED ADD A, 1
IFG A, I
    SET PC, PC ; we're done
SET PC, LOOP

:SCAN IFL A, I
	OUT A
SET B, A
:LOOP2 SET [B+3000], 1
ADD B, A
IFL B, I
	SET PC, LOOP2
SET PC, SCANNED
\end{lstlisting}
\end{minipage}

\begin{lstlisting}[caption={tri100.reg}]
SET C, 100 ; limit
SET I, 1
:LOOP SET J, 0
SET B, 0
:LOOP2 ADD J, 1
ADD B, J
IFL J, I
  SET PC, LOOP2
:BREAK OUT B
ADD I, 1
IFL I, C
  SET PC, LOOP
\end{lstlisting}
