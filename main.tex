\documentclass[11pt,a4paper,twoside,notitlepage]{report}
\usepackage[british]{babel} % British hyphenation patterns, etc.
\usepackage{csquotes}
\usepackage[T1]{fontenc}
\usepackage[a4paper]{geometry}
\usepackage{listings}
\usepackage{titling}
\usepackage[style=ieee,backend=biber]{biblatex}

\addbibresource{references.bib}

% Titlepage formatting
\pretitle{\begin{center}\Huge\bfseries}
\posttitle{\par\end{center}\vspace{0.5em}}
\preauthor{\begin{center}\Large\ttfamily}
\postauthor{\par\large\supervisor\end{center}}
\predate{\par\large\centering}
\postdate{\par\wordcount\par}

\immediate\write18{texcount \jobname.tex -inc -1 -sum -out=wordcount.txt}
\newcommand\wordcount{\small(\input{wordcount.txt}words, counted by \ttfamily{texcount})}
%texcount -sub=section \jobname.tex  | grep "Section" | sed -e 's/+.*//' | sed
%-n \thesection p > 'count.txt'

\title{Binary Code Translation from Register to Stack based code}
\author{Charles Pigott}
\date{DRAFT \today}
\newcommand\supervisor{Supervisor: Christopher Crispin-Bailey}

\begin{document}

\maketitle
\thispagestyle{empty}

%TC:group abstract 0 0
\begin{abstract}
	This project will require a strong affinity for machine level programming \&
	architecture understanding. Programming is more than likely to be undertaken
	in Java or C, depending upon the implementation platform. Competence in all
	of these areas would be a benefit.

	Transmeta are a US based company, who recently surprised the computer
	architecture world by developing a chip which dynamically translates Intel
	binary code into machine language instructions for another (highly optimised
	RISC) CPU core.  In doing so it allows Intel code to execute in real-time
	without recompilation, but using a (claimed) much more power efficient
	processor architecture.

	This project attempts to explore something related to the above experiment,
	however the focus will be on translating code into potentially parallel
	sections using a technique called microthreading. Choosing a relatively
	simple register machine for which complier, assembler, and simulator (or
	processor) are readily available would be a sensible approach. The project
	will involve writing a translator algorithm in high level code, but if time
	allows the translator can be –recoded in low-level code, or at least
	optimised to be as streamlined as possible (the speed of translation is a
	critical factor).
\end{abstract}

\listoffigures
\listoftables
\renewcommand*{\lstlistlistingname}{List of Listings}
\lstlistoflistings%

\cleardoublepage%
%\part{Preliminaries}

Blah blah blah\cite{GotoHarmful}

\printbibliography%

\end{document}
