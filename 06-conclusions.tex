\chapter{Conclusions}\label{ch:conclusions}
This project set out to show that it was possible to get performance
improvements out of stack machine code generated from code intended for a
register-based architecture. Ultimately, the results for this project showed a
reduction in memory accesses from the base translation after applying
optimisations and decreases in program cost using conversion block caching. The
implementation was also designed in such a way that allows for further
optimisations and improvements to be made to the project and the software was
released on GitHub to facilitate this.

All requirements to the project as defined in Problem Analysis
(Chapter~\ref{ch:problemanalysis}) were evaluated
(Section~\ref{sec:testingevaluation}) and it was found that all key requirements
were completed and but there was only partial completion of one of the optional
requirements. Both register architectures had ``feature complete'' emulators
that were able to parse and run their respective assembly code. Stack code is
able to be generated from each register instruction, with the ability to
translate whole programs, one block at a time. Through the use of commandline
flags, two different optimisation levels can be used --- peephole \&
stack-scheduling --- which reduce both instruction count and the number of
memory accesses compared to the unoptimised output.  Finally each converted
block was able to be cached, saving effort on converting it multiple times when
looped over.

The results of the tests vary depending on the program, but they generally
result around a 10\% decrease in raw stack instruction count and up to 25\%
decrease in number of memory accesses, importantly showing that stack machines
are not inherently less efficient at dealing with data accesses than register
machines. More significant and varied results were seen with conversion block
caching, with results varying from no improvement at all to up to 60\% decrease
in relative program cost based on the estimated memory model.

It's important to note the number of optional extra requirements that did not
get implemented. There were no real architectural issues involved with
implementing these requirements, rather they were only not completed due to time
constraints. Much of the work for this project was pushed towards the end of
allotted time rather than spread more evenly over the course of the year. If
this had been done, especially getting the literature review done in the first
few months, would've meant more time for the implementation stage and so more
features could've been implemented.

In conclusion, this project was able to show that there were several
improvements to be found in stack code generated from code intended for a
register machine.

\section{Further work}\label{sec:furtherwork}
There are several further steps that could be applied to this project.

First and foremost would be to implement further optimisations, such as the
inter-block algorithm by Bailey~\cite{Bailey2000Inter} and the refinements done
by Shannon~\cite{Shannon2006AC} show great potential for further improvements
especially with the tight loops of the programs tested in this project. These
optimisations would only improve the generated stack code even further and its
relative performance to register machines. Beyond that, it's unlikely that all
types of optimisation for this type of stack code have been found and further
algorithmic improvements are possible. As such, it's possible that further
improvements, with the right stack machine, may show stack machines as the more
efficient type of processor.

There's also further scope for improving the instruction block caching system.
Other than improving how blocks are divided up, making use of Transmeta's
`translation threshold' method could prove more effective for reducing the
relative cost of running a program. Further work could also be put into using a
more accurate memory model, as the results of the program cost
(Figure~\ref{fig:progcost}) from the one used in this project are only drawn
from an approximation taken from a flat memory model and an estimated cost for
translation. Transmeta ran into difficulties with their VLIW RISC-style
processor, with their CMS being underutilised in all but specific benchmarks,
but perhaps with a stack machine as a base processor it could achieve better
results.

Other options would be to more properly prove that the implementation of the
emulators and the conversion routines are correct. Formal methods to
mathematically prove correctness could be used. This could be taken further to
automatically generate test programs that try to break the conversion or the
emulators, generally referred to as ``fuzz testing''.

An ambitious goal would be to implement the conversion and optimisation routines
in the target stack architecture. This would mean that the the conversion could
be implemented natively on a processor capable of running the stack program.
Such a program would be very complicated so a stack language capable of
high-level features such as functions would be necessary to make this goal
feasible.
