\chapter{Conclusions}\label{ch:conclusions}
In conclusion, this project was successful in its aim of finding performance
improvements in generated stack code from the na{\"\i}ve model of doing so. A
variety of optimisation techniques can be applied to the generated code to
both reduce raw instruction count and reduce memory usages. The results vary
depending on the program, but the results for this project resulted in around a
10\% decrease in raw stack instruction count and up to 25\% decrease in number
of memory accesses. More significant results were seen with conversion block
caching, seeing an up to 60\% decrease in relative program cost based on an
estimated memory model.

\section{Further work}\label{sec:furtherwork}
There is much further work that can be applied to this project.

First and foremost would be to implement further optimisations, such as the
inter-block algorithm by Bailey and the refinements done by Shannon show great
potential for further improvements especially with the tight loops of the
programs tested in this project.

There's also further scope for improving the instruction block caching system.
Other than improving how blocks are divided up, making use of Transmeta's
`translation threshold' method could prove more effective for reducing the
relative cost of running a program.

Other options would be to more properly prove that the implementation of the
emulators, and the conversion routines is correct. Formal methods to
mathematically prove correctness could even be used if so inclined. This could
be taken further to automatically generate test programs that try to break the
conversion or the emulators, generally referred to as ``fuzz testing''

