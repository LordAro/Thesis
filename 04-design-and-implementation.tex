\chapter{Design and Implementation}\label{ch:designimplementation}

An early decision to be made was to write emulators for the register \& stack
machines chosen. Toolchains for `obscure' architectures such as the ones likely
to be chosen tend to be rather limited in nature and either out-of-date, broken
or both. Writing emulators for the architectures creates extra programming
work, but means that the choices of programming language and style are much
greater.

It was then decided that to simplify the design, the program would be interpret
assembly source code of the register \& stack architectures instead of running
the compiled binary. Doing it this way is only an abstraction over actually
reading the binaries, and skips out having to compile the source code and then
decode it again. Relatively speaking, it would not be difficult to make a
program that does this but was deemed unnecessary for the core part of this
project, converting a register-based instruction set to a stack-based one.

\section{Programming language choice}
C++ is a systems programming language that is well known for it's hugely
powerful templates for generic programming, its ability to use multiparadigm
styles of programming and its high speed, with significant work being put into
optimising compilers for the language. Using object-orientated-programming, the
conversion routines can be built on top of one of the emulators, with little
duplication of code.

C++ does have some disadvantages however. Its (relatively) low-level nature and
power makes it easy to make errors in programming and logic, and its compiler
errors for templates are infamous for taking up many pages. In fact, C++'s own
creator has been quoted as saying ``C makes it easy to shoot yourself in the
foot; C++ makes it harder, but when you do it blows your whole leg off'' when
comparing C++ to its parent language C. That said, modern C++ has been making
big improvements to the language since 2011, with smart pointers, lambdas \&
type deduction fixing many of the language's major criticisms.

Some thought was put into other higher-level languages, such as Python, but as
these languages are interpreted instead of compiled, they incur an inherent
speed penalty by comparison which with a speed sensitive project such as this
where the emulators need to complete executing instructions in a specified
amount of time is not a thing that is desirable to be worrying about.

With all this in mind, it was decided that existing familiarity with the C++
language was important as were the benefits of not having to worry about the
speed of the emulators themselves for development outweighed the disadvantages.

\section{Architecture choices}
An immediate shortlist of register architectures was drawn up, of the Z80, the
Picoblaze, and the DCPU-16.

The Zilog Z80 is a microprocessor introduced in 1976. It is 8-bit based with the
capability to address 64KB of memory by means of combining its registers. An
extremely popular processor during the 1980s, it is still produced to this day
for uses in embedded systems. It has its own assembly language that has both
register and stack elements. Xilinx's Picoblaze is a `soft processor core',
meaning that isn't manufactured specially, rather is fabricated on an FPGA
(field-programmable gate array) by the user. It uses an 8-bit RISC architecture,
which makes it very simple to fabricate and run. The DCPU-16 isn't actually a
real processor, and has never been produced. It is the invention of Markus
Persson to be emulated as part of a game he was creating that never ended up
getting finished, but there was significant interest in it, which means that
there was many programs and emulators produced for the processor regardless.
It's a 16-bit processor that has modularity in mind in the architecture --- the
computer would `connect' to peripherals to do its IO instead of a more common
console output interface.

\noindent\begin{minipage}{0.5\textwidth}
\begin{lstlisting}[caption={Z80 ASM}]
      LD a, 0
      LD (CURCOL), a
      LD (CURROW), a
      LD hl, text
      B_CALL(_PutS)
      RET
text:
      .db "Hello, world!", 0
\end{lstlisting}%
\end{minipage}%
\noindent\begin{minipage}{0.5\textwidth}
\begin{lstlisting}[caption={Picoblaze ASM}]
module hello_world;

initial begin
  $display ("Hello, world!");
  #10 $finish;
end

endmodule
\end{lstlisting}%
\end{minipage}

\noindent\begin{minipage}{\linewidth} %prevent page break
\begin{lstlisting}[caption={DCPU-16 ASM}]
; Attach screen
SET A, 0
SET B, vram
HWI 0

SET J, 0

:loop
SET I, vram
ADD I, J
ADD [I], 0x2000
ADD J, 1
IFN J, 12
    SET PC, loop

:crash
SET PC, crash

:vram
DAT "Hello, world!", 0
\end{lstlisting}%
\end{minipage}

As the Picoblaze is a soft-processor, it is separated into its ASM and Verilog
components and you have to connect the processor's `wires' together yourself. It
was decided that this would be too much boilerplate to be worth the effort, so
the Picoblaze was discounted. After some initial emulator implementation for the
Z80 processor, it was found that its ability to combine registers were not easy
to implement in a good way in C++, so after some thought it was decided to
continue with the DCPU-16 assembly language, after cutting it down to get rid of
the peripherals part of the language which isn't necessary for this project.

In terms of stack architectures, Forth, JVM bytecode, and J5 were shortlisted.

Forth is a very early programming languages that dates from 1970, when stack
machines were still common in everyday computing. It's a very simple compiled
language that still finds uses today in embedded systems due to how easy it is
to port to new systems, and its low memory overhead. J5 is a teaching stack
language to introduce people to stack machines and their respective languages
and it is very simple in that regard, drawing on ideas from various different
actual stack languages.

\noindent\begin{minipage}{0.5\textwidth}
\begin{lstlisting}[caption={Forth ASM}]
CR .( Hello, world!)
\end{lstlisting}
\end{minipage}%
\noindent\begin{minipage}{0.5\textwidth}
\begin{lstlisting}[caption={J5 ASM}]
OUT "Hello, world!"
\end{lstlisting}
\end{minipage}

\section{Emulator implementation}
Early in the implementation, it was decided to cut down on the instructions that
the emulators, as they were either esoteric or difficult to implement while
being irrelevant to this project. In particular, none of the special opcodes for
the DCPU-16 (see Appendix~\ref{ch:dcpuspec}) were implemented, and neither were
the \lstinline{PUSH} or \lstinline{POP} identifiers, as they are used to
implement a `reverse stack' starting at the top of the memory, 0xffff. Since
this project tests converting register code to stack code, it doesn't make much
sense to have most of a stack in the register architecture. This does exclude
the possibility of having a call stack, with procedures or functions, but this
is an acceptable compromise.

Similarly, the J5 (see Appendix~\ref{ch:j5spec}) includes \lstinline{CALL nn} \&
\lstinline{RETURN} instructions, so for feature parity it was decided to exclude
these as well. The J5 also has a \lstinline{SSET n} function, which is intended
as a fast version of \lstinline{SET nn} for small numbers. Since this is an
emulator, this can be replaced with \lstinline{SET} quite trivially. However, if
it were included, it would be a good target for optimisation, as since the
operand is constant, it would be trivial to replae the \lstinline{SET}
instruction with \lstinline{SSET} where appropriate.

\subsection{Algorithms used}
The general structure for both emulators is fairly similar. First the input
file is read into memory. It is then iterated over, line-by-line, and tokenised
into an intermediary representation, which serves to validate the input syntax.
This can be approximated with the following psuedocode for the register
architecture:

\begin{lstlisting}[caption={Tokenising algorithm for the DCPU-16}]
function tokenise(filepath: string): instruction
  load file from filepath
  For line in file do
    instruction;
    split line into words # words separated by whitespace
    word = words[0]
    If word is a label # is prefixed by a ':'
      instruction.label = first word
      word++
    If word is not an opcode
      return error "not a valid instruction"
    instruction.code = word
    word++

    # opcodes take a variable number of operands
    If number of operands for instruction.code != remaining_words
      return error "invalid instruction" + instruction.opcode
    instruction.operands = remaining words
    return instruction
  end
\end{lstlisting}

It was quite difficult to determine the type of the operands from the source.
C++ in particular made it difficult due to its type system. The eventual
solution was to use a variant (typesafe union) type to store integers (for
literals), register references, and strings for everything else. The
``everything else'' proved to be quite complicated as it included not only
labels and string literals, but memory address lookups as well (e.g.
\lstinline{SET [2000], 42}). For executing purposes it was possible to check if
the string started and ended with `[' \& `]' respectively, and getting the value
of the operand inside. This was further complicated as the DCPU-16 has the
ability to use compound expressions inside the memory address brackets, e.g.
\lstinline{[2000+I]} where the contents of the `I' register is added to 2000 to
get the memory address referred to.

A similar solution was done with the stack
emulator.

For running the emulators themselves, they provided a \lstinline{run} function,
which is passed the tokenised program to run (along with any flags such as
verbosity of output). It then initialises the machine's state and runs through
each instruction in the program until terminated or it reaches the end.

\begin{lstlisting}[caption={Running a DCPU-16 program}]
procedure run(program: Program, flags..)
  set registers to 0
  program_counter = 0
  while program_counter < sizeof(program)
    run_instruction(program[program_counter])
    program_counter++
  end
\end{lstlisting}

The above is an overly simplistic view of the actual function though, which has
the ability to run at a certain clock speed (to mimic the speed the actual
processor would process instructions) or just as fast as the emulator can run,
which is more useful for testing purposes. There's also a verbosity flag which
dumps the state of the machine, for debugging purposes.

The function loop also has to deal with the effects of the conditional
expressions. For the DCPU-16 this is the \lstinline{IFx} statements where they
only perform the next expression if the condition is true, whereas the the J5
has some test instructions (e.g. \lstinline{TEQ}) and \lstinline{BRZERO}
instruction which conditionally branches on whether the test instruction set the
appropriate flag. The notable thing here is that the test instruction can be
executed long before the branch is actually taken, which allows for more freedom
over branching. It also has \lstinline{BRANCH} for unconditionally branching.

\section{Conversion routines}
Initially, the conversion was done all at once --- the program would take in a
register program and output a stack program that could be run by the stack
interpreter. This served well as an initial step, but to get to point of caching
blocks of code, the implementation had to be changed first to converting one
instruction at a time and then executing it as the stack machine all at once, to
converting blocks of instructions and executing them. The process for this can
be seen in Listing~\ref{convertcache}

\begin{lstlisting}[label=convertcache,caption={Converted instruction
caching},float]
cache: map positions to stack code;
function convert_instructions(start_pos: instruction)
  If start_pos not in cache
    last_pos = find next label
    cache[start_pos] = convert_instructions(start_pos, last_pos)
  return cache[start_pos]
\end{lstlisting}

Table~\ref{tab:conversionexs} is a table that shows some examples of the stack
code produced for some register instructions.  By way of explanation, for the
\lstinline{ADD A, B} row the code snippet first loads register A (represented
by 8191) onto the stack, loading register B (8190) onto the stack, adding them
together, and storing the result back into A. The DCPU-16's \lstinline{IFx}
statements are more complex to convert as the J5 does not have a direct
equivalent. The \lstinline{IFx} instruction only executes the next instruction
if the condition (e.g. \lstinline{B != A} for \lstinline{IFN B, A}) evaluates to
true. For the J5, its only conditional branching mechanism is \lstinline{BRZERO}
and a series of `test' instructions, that have to be executed before hand and
set a flag to signify whether its test is true or not. So, the conversion
routine needs to load the operands for the statement onto the stack, test the
operands on the stack, the drop the operands (so not to leave them ``dangling''
on the stack regardless of the condition result), then \lstinline{BRZERO} with
an operand that's equal to the generated length of the next instruction.
Implementation-wise, it's necessary to leave a temporary value for the operand
initially, then go back and change it once the next instruction has been
converted.

The implementation of this is particularly nice as the functions which load the
instruction operands onto the stack (as in, e.g., \lstinline{ADD B, A}) were
able to use one another, with minimal code duplication. For example, loading a
register value onto the stack is just the same as loading an address onto the
stack but with an extra \lstinline{LOAD} instruction following it. These segment
functions allowed operands to be loaded onto the stack with no additional
side effects. A complicated example of this can be seen with the
\lstinline{SET A, [3000+I]} in Table~\ref{tab:conversionexs} that uses the
DCPU-16's ability to have arithmetic operations inside memory addresses.

\begin{table}
\caption{Instruction conversion examples}
\begin{tabular}{l l}\label{tab:conversionexs}
Register instruction & Converted stack code \\ \toprule
\begin{lstlisting}
SET A, 5000
\end{lstlisting} &
\begin{lstlisting}
SET 5000
SET 8191
STORE
\end{lstlisting} \\ \midrule
\begin{lstlisting}
ADD A, B
\end{lstlisting} &
\begin{lstlisting}
SET 8191
LOAD
SET 8190
LOAD
ADD
SET 8191
STORE
\end{lstlisting} \\ \midrule
\begin{lstlisting}
SET PC, LOOP
\end{lstlisting} &
\begin{lstlisting}
BRANCH LOOP
\end{lstlisting} \\ \midrule
\begin{lstlisting}
IFN A, 0
SET PC, LOOP
\end{lstlisting} &
\begin{lstlisting}
SET 8191
LOAD
SET 0
TEQ
DROP
DROP
BRZERO 2
BRANCH LOOP
\end{lstlisting} \\ \midrule
\begin{lstlisting}
SET A, [3000+I]
\end{lstlisting} &
\begin{lstlisting}
SET 3000
SET 8185
LOAD
ADD
LOAD
SET 8191
STORE
\end{lstlisting} \\
\end{tabular}
\end{table}


\section{Optimisation}
Optimisation was done on the stack code generated by blocks of register
instructions.

\subsection{Peephole optimisation}
A number of peephole optimisations were implemented as part of the optimisation
process. A list of them can be seen in Table~\ref{tab:peepholeex}. While it has
some affect just on its own, typically around the `edges' of generated stack
code, it tends to have greater effect following a stack scheduling pass.

The optimiser iterates over the program (which is again separated into regions
by label) in blocks of a certain size that's determined by the individual
optimisation. When it finds a block that matches the pattern that it's looking
for, it replaces that block as defined by the optimisation.

% Oh dear.
\begin{table}
\caption{Peephole optimisation examples}
\begin{tabularx}{\linewidth}{l l X}\label{tab:peepholeex}
Original stack code & Optimised stack code & Notes \\ \toprule
\begin{lstlisting}^^J
SET 1^^J
ADD
\end{lstlisting} &
\begin{lstlisting}^^J
INC
\end{lstlisting} &
Take advantage of the J5's {\lstinline!INC!} instruction \\ \midrule
\begin{lstlisting}^^J
SET 1^^J
SUB
\end{lstlisting} &
\begin{lstlisting}^^J
DEC
\end{lstlisting} &
Same as above, but for subtraction \\ \midrule
\begin{lstlisting}^^J
SET 0^^J
TEQ^^J
DROP
\end{lstlisting} &
\begin{lstlisting}^^J
TSZ
\end{lstlisting} &
Generated with an {\lstinline!IFN, x, 0!} statement. Instead take advantage of
the {\lstinline!TSZ!} instruction of the J5 \\ \midrule
\begin{lstlisting}^^J
SET x^^J
STORE^^J
SET x^^J
LOAD
\end{lstlisting} &
\begin{lstlisting}^^J
DUP^^J
SET x^^J
STORE
\end{lstlisting} &
A store followed by an immediate load of the same value. Instead, duplicate the
stored value and just store (since it might be used elsewhere) \\ \midrule
\begin{lstlisting}^^J
DUP^^J
SWAP
\end{lstlisting} &
\begin{lstlisting}^^J
DUP
\end{lstlisting} &
Removes redundant swap \\ \midrule
\begin{lstlisting}^^J
SWAP^^J
SWAP
\end{lstlisting} &
\begin{lstlisting}^^J
<NOP>
\end{lstlisting} &
Removes redundant swaps \\ \midrule
\begin{lstlisting}^^J
SET x^^J
DROP
\end{lstlisting} &
\begin{lstlisting}^^J
<NOP>
\end{lstlisting} &
Removes redundant swaps \\ \bottomrule
\end{tabularx}
\end{table}

\subsection{Stack scheduling}
Stack scheduling is relatively simple to implement and is described in
pseudocode by Listing~\ref{lst:schedule}.

\begin{lstlisting}[caption={Stack scheduling
implementation},float,label=lst:schedule]

pairs = []

function get_stack_depth(pos)
  return number of pairs such that p.i < pos and pos < p.j

For ins = program.begin to program.end
  if ins == SET and ins + 1 == STORE
    For ins2 = ins + 2 to program.end
      if ins2 == SET and ins2 + 1 == LOAD
        # found a match
        pairs.append({ins, ins2})
        break ; don't include the same values more than once
    end
end
sort pairs by distance (j-i) # so we apply tightest loops first
For (i, j) in pairs
  stack_depth = get_stack_depth(i)
  if stack_depth > 2 or get_stack_depth(j) > 2
    continue ; too deep in the stack to bring back to the top

  if stack_depth == 0
    ins = DUP
  else if stack_depth == 1
    ins = TUCK2
  else if stack_depth == 2
    ins = TUCK3

  remove instructions at positions j and j+1
  insert ins before position i

  stack difference = net stack depth change between i and j
  if stack difference == 0
    # no op
  else if stack difference == 1
    insert SWAP after position j
  else if stack difference == 2
    insert RSD3 after position j

  For p in remaining pairs
    adjust i, j according to insertion/deletion change
  end
end
\end{lstlisting}

The key to the instruction that needs to be inserted at the ``reuse'' point to
bring the instruction to the top of the stack depends on how many values have
been added by instructions between its store point and its reuse point. This can
be determined statically as all instructions have a known behaviour (whether
they remove or add values to the stack).
