\chapter{Design and Implementation}\label{ch:designimplementation}

\section{Emulator implementation}
Both the register and stack emulators (plus the conversion ``emulator'') are
written to be part of the same program, with the different functionality
controlled with commandline flags. This method enables the emulators to share
functionality where appropriate, especially for the conversion `machine' which
is built directly on top of the stack machine using object orientated
programming techniques.

Early in the implementation, it was decided to cut down on the instructions that
the emulators, as they were either esoteric or difficult to implement while
being irrelevant to this project. In particular, none of the special opcodes for
the DCPU-16 (see Appendix~\ref{ch:dcpuspec}) are implemented, and neither are
the \texttt{PUSH} or \texttt{POP} instructions, as they are used to implement a
`reverse stack' starting at the top of the memory, 0xffff. Since this project
tests converting register code to stack code, it does not make much sense to
have most of a stack in the register architecture. This does exclude the
possibility of having a call stack, with procedures or functions, but this is an
acceptable compromise as they can be emulated by branching instructions instead.

Similarly, the J5 (see Appendix~\ref{ch:j5spec}) includes \texttt{CALL nn} \&
\texttt{RETURN} instructions, so for feature parity these were excluded as well.
The J5 also has a \texttt{SSET n} function, which is intended as a fast version
of \texttt{SET nn} for small numbers. Since this is an emulator, this is
replaced with \texttt{SET} quite trivially. However, if it were included, it
would be a good target for optimisation, as the operand is constant, it would be
trivial to replace the \texttt{SET} instruction with a \texttt{SSET} in the
appropriate places.

\subsection{Algorithms used}
The general structure for both emulators is fairly similar. First the input
file is read into memory. It is then iterated over, line-by-line, and tokenised
into an intermediary representation, which serves to validate the input syntax.
This can be approximated with the following pseudocode for the register
architecture:

\noindent\begin{minipage}{\linewidth}
\begin{lstlisting}[caption={Tokenising algorithm for the DCPU-16}]
function tokenise(filepath: string): instruction
  load file from filepath
  For line in file do
    instruction;
    split line into words # words separated by whitespace
    word = words[0]
    If word is a label # is prefixed by a ':'
      instruction.label = first word
      word++
    If word is not an opcode
      return error "not a valid instruction"
    instruction.code = word
    word++

    # opcodes take a variable number of operands
    If number of operands for instruction.code != remaining_words
      return error "invalid instruction" + instruction.opcode
    instruction.operands = remaining words
    return instruction
  end
\end{lstlisting}
\end{minipage}

Determining the type of the operands from the source proved difficult.
C++ in particular made it difficult due to its type system. The eventual
solution uses a variant (typesafe union) type to store integers (for
literals), register references, and strings for everything else. This
necessitated the use of a third-party library, Boost, as C++ does not have such
a type in its standard library. Although it does gain one in C++17, the latest
standard that as of 2017 is not yet fully included in a stable release of any of
the common compilers. Boost is a large and complicated library, but if usage is
limited to only required features it does not increase maintenance costs
significantly. Using strings for the `everything else' part of the operand
proves quite complicated as it includes not only labels and string
literals, but memory address lookups as well (e.g.\ \texttt{SET [2000], 42}).
For executing purposes it is possible to check if the string starts and ends
with \texttt{\lbrack} and \texttt{\rbrack} respectively, and getting the value
of the operand inside. This is further complicated as the DCPU-16 has the
ability to use compound expressions inside the memory address brackets, e.g.
\texttt{[2000+I]} where the contents of the `I' register is added to 2000 to
get the memory address referred to. A similar solution is made for the stack
emulator.

For running the emulators themselves, their class provides a \texttt{run}
function, which is passed the tokenised program to run (along with any flags
such as verbosity of output). It then initialises the machine's state and runs
through each instruction in the program until terminated or it reaches the end.

\begin{lstlisting}[caption={Running a DCPU-16 program}]
procedure run(program: Program, flags..)
  set registers to 0
  program_counter = 0
  while program_counter < sizeof(program)
    run_instruction(program[program_counter])
    program_counter++
  end
\end{lstlisting}

The above is a simple view of the actual function though, which has the ability
to run at a certain clock speed (to mimic the speed the actual processor would
process instructions) or just as fast as the emulator can run, which is more
useful for testing purposes. There is also a verbosity flag which dumps the
state of the machine, for debugging purposes.

For each parsed instruction, a member function for the class of the relevant
emulator has been written. To link each of these methods to their respective
instruction, it is best to use a \texttt{std::map} as provided by the standard
library. In many places, the functions are very short, only consisting of a
single line of code, so the code makes use of C++'s lambdas. This means, for
example, that the implementation for the J5's \texttt{INC} instruction only
consists of the lambda \texttt{[](machine~*m)\{m->stack.top()++;\}} instead of
referring to a function pointer such as \lstinline{&machine::inc_func} in the
class' function map, which avoids all the extra code overhead of having to fully
declare an extra method. Where the instruction implementation functions were
beyond a line, the usual function pointer style has been kept to help keep the
code readable. Taking this, many of the instructions follow a similar pattern
(for example, \texttt{ADD} and \texttt{SUB} from the J5 both operate on the top
two items on the stack). As the code for these implementations would have
otherwise been duplicated several times (where only the exact operation
differs), a helper function has been created to deal with these. This results in
the lambda \texttt{[](machine~*m)\{m->binop\_func([](int~a,~int~b)\{return
a+b;\});\}} which even though works as expected, it is decidedly unreadable for
anyone looking through the code for themselves.

The function loop also has to deal with the effects of the conditional
expressions. For the DCPU-16 this is the \texttt{IFx} statements where they only
perform the next expression if the condition is true, whereas the J5 has some
test instructions (e.g. \texttt{TEQ}) and \texttt{BRZERO} instruction which
conditionally branches on whether the test instruction set the appropriate flag.
The notable thing here is that the test instruction can be executed long before
the branch is actually taken, which allows for more freedom over branching. It
also has \texttt{BRANCH} for unconditionally branching.

\section{Conversion routines}

Initially, implementation of the conversion was to do it all at once --- the
program would take in a register program and output a stack program that could
be run by the stack interpreter. This served well as an initial step, but to get
to the point of caching blocks of code, the implementation was to be changed
first to converting one instruction at a time and then executing it as the stack
machine all at once, to converting blocks of instructions and executing them.
The process for this can be seen in Listing~\ref{lst:convertcache}

\begin{lstlisting}[label={lst:convertcache},caption={Converted instruction
caching},float]
cache: map positions to stack code;
function convert_instructions(start_pos: instruction)
  If start_pos not in cache
    last_pos = find next label
    cache[start_pos] = convert_instructions(start_pos, last_pos)
  return cache[start_pos]
\end{lstlisting}

Table~\ref{tab:conversionexs} is a table that shows some examples of the stack
code produced for some register instructions. By way of explanation, for the
\texttt{ADD A, B} row the code snippet first loads register A (represented by
8191) onto the stack, loading register B (8190) onto the stack, adding them
together, and storing the result back into A. The DCPU-16's \texttt{IFx}
statements are more complex to convert as the J5 does not have a direct
equivalent. The \texttt{IFx} instruction only executes the next instruction if
the condition (e.g. \texttt{B != A} for \texttt{IFN B, A}) evaluates to true.
For the J5, the only conditional branching mechanism is \texttt{BRZERO} and a
series of `test' instructions, that have to be executed before hand and set a
flag to signify whether its test is true or not. So, the conversion routine
needs to load the operands for the statement onto the stack, test the operands
on the stack, the drop the operands (so not to leave them `dangling' on the
stack regardless of the condition result), then \texttt{BRZERO} with an operand
that is equal to the generated length of the next instruction. With regard to
implementation, it was necessary to leave a temporary value for the operand
initially, then go back and change it once the next instruction had been
converted.

The implementation of this is particularly nice as the functions which load the
instruction operands onto the stack (as in, e.g., \texttt{ADD B, A}) were able
to use one another, with minimal code duplication. For example, loading a
register value onto the stack is just the same as loading an address onto the
stack but with an extra \texttt{LOAD} instruction following it. These segment
functions allow operands to be loaded onto the stack with no additional side
effects. A complicated example of this can be seen with the \texttt{SET A,
[3000+I]} in Table~\ref{tab:conversionexs} that uses the DCPU-16's ability to
have arithmetic operations inside memory addresses.

\begin{table}
\caption{Instruction conversion examples}
\begin{tabular}{l l}\label{tab:conversionexs}
Register instruction & Converted stack code \\ \toprule
\begin{lstlisting}
SET A, 5000
\end{lstlisting} &
\begin{lstlisting}
SET 5000
SET 8191
STORE
\end{lstlisting} \\ \midrule
\begin{lstlisting}
ADD A, B
\end{lstlisting} &
\begin{lstlisting}
SET 8191
LOAD
SET 8190
LOAD
ADD
SET 8191
STORE
\end{lstlisting} \\ \midrule
\begin{lstlisting}
SET PC, LOOP
\end{lstlisting} &
\begin{lstlisting}
BRANCH LOOP
\end{lstlisting} \\ \midrule
\begin{lstlisting}
IFN A, 0
SET PC, LOOP
\end{lstlisting} &
\begin{lstlisting}
SET 8191
LOAD
SET 0
TEQ
DROP
DROP
BRZERO 2
BRANCH LOOP
\end{lstlisting} \\ \midrule
\begin{lstlisting}
SET A, [3000+I]
\end{lstlisting} &
\begin{lstlisting}
SET 3000
SET 8185
LOAD
ADD
LOAD
SET 8191
STORE
\end{lstlisting} \\
\end{tabular}
\end{table}


\section{Optimisation}
Optimisation is done on the stack code generated by blocks of register
instructions.

\subsection{Peephole optimisation}
A number of peephole optimisations are implemented as part of the optimisation
process. A list of them can be seen in Table~\ref{tab:peepholeex}. While it has
some effect on its own, typically around the `edges' of generated stack
code, it tends to have greater effect following a stack scheduling pass.

The optimiser iterates over the program (which is again separated into regions
by label) in blocks of a certain size that is determined by the individual
optimisation. When it finds a block that matches the pattern that it is looking
for, it replaces that block as defined by the optimisation.

% Oh dear.
\noindent\begin{table}
\caption{Peephole optimisation examples}
\begin{tabularx}{\linewidth}{L L X}\label{tab:peepholeex}
  \bigskip\normalfont{Original\newline stack code} &
  \bigskip\normalfont{Optimised\newline stack code} &
  \bigskip Notes \\ \toprule
SET 1\newline
ADD
&
INC
&
Take advantage of the J5's \texttt{INC} instruction \\ \midrule
SET 1\newline
SUB
&
DEC
&
Same as above, but for subtraction \\ \midrule
SET 0\newline
TEQ\newline
DROP
&
TSZ
&
Generated with an \texttt{IFN, x, 0} statement. Instead take advantage of the
\texttt{TSZ} instruction of the J5 \\ \midrule
SET x\newline
STORE\newline
SET x\newline
LOAD
&
DUP\newline
SET x\newline
STORE
&
A store followed by an immediate load of the same value. Instead, duplicate the
stored value and just store (since it might be used elsewhere) \\ \midrule
DUP\newline
SWAP
&
DUP
&
Removes redundant swap \\ \midrule
SWAP\newline
SWAP
&
<NOP>
&
Removes redundant swaps \\ \midrule
SET x\newline
DROP
&
<NOP>
&
Removes unused stack usage \\ \bottomrule
\end{tabularx}
\end{table}

\subsection{Stack scheduling}
Stack scheduling is simple to implement and is described in pseudocode by
Listing~\ref{lst:schedule}.

\begin{lstlisting}[caption={Stack scheduling
implementation},float,label=lst:schedule]

pairs = []

function get_stack_depth(pos)
  return number of pairs such that p.i < pos and pos < p.j

For ins = program.begin to program.end
  if ins == SET and ins + 1 == STORE
    For ins2 = ins + 2 to program.end
      if ins2 == SET and ins2 + 1 == LOAD
        # found a match
        pairs.append({ins, ins2})
        break ; don't include the same values more than once
    end
end
sort pairs by distance (j-i) # so we apply tightest loops first
For (i, j) in pairs
  stack_depth = get_stack_depth(i)
  if stack_depth > 2 or get_stack_depth(j) > 2
    continue ; too deep in the stack to bring back to the top

  if stack_depth == 0
    ins = DUP
  else if stack_depth == 1
    ins = TUCK2
  else if stack_depth == 2
    ins = TUCK3

  remove instructions at positions j and j+1
  insert ins before position i

  stack difference = net stack depth change between i and j
  if stack difference == 0
    # no op
  else if stack difference == 1
    insert SWAP after position j
  else if stack difference == 2
    insert RSD3 after position j

  For p in remaining pairs
    adjust i, j according to insertion/deletion change
  end
end
\end{lstlisting}

The key to which instruction needs to be inserted at the `reuse' point to bring
the inserted instruction to the top of the stack depends on how many values have
been added by instructions between its store point and its reuse point. This can
be determined statically as all instructions have a known behaviour (whether
they remove or add values to the stack).

An example of what this algorithm produces is listed in
Table~\ref{tab:scheduleex}. It is worth noting that even though the stack code
produced by the stack scheduling is only 3 instructions shorter, it only
contains 6 memory read/writes, compared to 10 for the initial unoptimised stack
code which is a 40\% reduction.

\begin{table}
\caption{Stack scheduling example}
\begin{tabular}{c c c}\label{tab:scheduleex}
Original register code & Initial stack code & Stack scheduled code \\ \toprule
\begin{lstlisting}
; b = a + c
SET A, 23
SET C, 7
ADD X, A
ADD X, C
SET B, X
\end{lstlisting} &
\begin{lstlisting}
SET 23
SET 8191
STORE
SET 7
SET 8189
STORE
SET 8188
LOAD
SET 8191
LOAD
ADD
SET 8188
STORE
SET 8188
LOAD
SET 8189
LOAD
ADD
SET 8188
STORE
SET 8188
LOAD
SET 8190
STORE
\end{lstlisting} &
\begin{lstlisting}
SET 23
DUP
SET 8191
STORE
SET 7
TUCK2
SET 8189
STORE
SET 8188
LOAD
SWAP
ADD
DUP
SET 8188
STORE
ADD
DUP
SET 8188
STORE
SET 8190
STORE
\end{lstlisting} \\
\end{tabular}
\end{table}

Due to the na{\"\i}vity of how the blocks are determined (splitting on labels),
if a loop does not have any labels that segment the code that follows the end of
the loop, it will also be included in the block.  This causes issues with the
stack-scheduling optimiser as it may find pairs of \texttt{STORE}/\texttt{LOAD}
which are actually outside the block. For the purposes of this project, an
(unused) label at the actual logical end of the block.

\section{Overall structure}
The overall structure of how the translator performs is shown in
Figure~\ref{fig:implementationstructure}. The program is similar to Transmeta's
CMS structure (as seen in Figure~\ref{fig:cmsctrlflow}) with its translator and
associated cache system along with how it iterates and refers to the translated
code, but lacks the translation threshold and the ability to rollback and
interpret the instruction in the register language. While this implementation
can interpret the register code, it cannot switch between that and translating
in a single run. This diagram also shows the control flow for the optimisation
patterns which Transmeta's diagram lacks. Similarly to the Transmeta diagram, it
also does not show the end condition of the Interpreter/Translator combination,
when the program counter reaches the end of the program, or an unrecoverable
fault occurs.

\begin{figure}
  \begin{tikzpicture}[node distance=3cm, on grid, auto, >=stealth, thick,
                      text centered, font=\small]
    \tikzstyle{process} = [rectangle, minimum width=3cm, minimum height=1cm,
                           draw=black, fill=white, text width=2.5cm]
    \tikzstyle{decision} = [diamond, minimum width=3cm, minimum height=1cm,
                            draw=black, fill=white, text width=2.5cm]

    \begin{scope}[on background layer]
      \fill[blue!60!,opacity=.3] (-4, 1) rectangle (9,-16);
      \fill[white] (-4, 1) rectangle (3, -7.5);
      \fill[blue!25!,opacity=.3] (-4, 1) rectangle (3, -7.5);
      \node[right] at (-4,.5) {\textbf{Interpreter}};
      \node[left] at (9,.5) {\textbf{Translator}};
    \end{scope}

    \node (start) [process, minimum width=2cm] {Start};
    \node (dec1)  [decision, below=of start] {Next region in cache?};
    \node (pro1)  [process, right=of dec1, xshift=3cm] {Translate Region};
    \node (pro2)  [process, below=of dec1] {Execute Translation from cache};
    \node (dec2)  [decision, below=of pro1] {Is optimisation\allowbreak enabled?};
    \node (pro3)  [process, below=of dec2] {Apply peephole\allowbreak optimisations};
    \node (dec3)  [decision, below=of pro3] {Is Koopman optimisation enabled?};
    \node (pro6)  [process, below=of pro2] {Store region in cache};
    \node (pro5)  [process, below=of pro6] {Apply peephole\allowbreak optimisations};
    \node (pro4)  [process, below=of pro5] {Apply stack-scheduling\allowbreak optimisations};
    \coordinate[right=of pro6] (invis);

    \draw [->] (start) -- (dec1);
    \draw [->] (dec1) -- node[anchor=north east] {no} (pro1);
    \draw [->] (dec1) -- node {yes} (pro2);
    \draw [->] (pro1) -- (dec2);
    \draw [->] (dec2) -- node {yes} (pro3);
    \draw [->] (invis) -- (pro6);
    \draw (dec2.south west) -- node[anchor=south east] {no} (invis);
    \draw (dec3.north west) -- node[anchor=north east] {no} (invis);
    \draw [->] (pro3) -- (dec3);
    \draw [->] (dec3) |- node {yes} (pro4);
    \draw [->] (pro4) -- (pro5);
    \draw [->] (pro5) -- (pro6);
    \draw [->] (pro6) -- (pro2);
    \draw [->] ([yshift=3mm] pro2.south east) |- ++(1, 0) |-
    node[yshift=-3mm, anchor=north east] {chain} ([yshift=-3mm] pro2.north east);
    \draw [->] (pro2.west) node[anchor=north east, text width=1.5cm] {no chain} -| (dec1.west);
  \end{tikzpicture}
  \caption{Implementation structure}\label{fig:implementationstructure}
\end{figure}


\section{Test implementations}

For getting results out, the programs have to be converted several times with
various different commandline flags for the register vs stack conversion and for
the different optimisation levels. For this a simple `test harness' has been
written in Python, which loops through all the programs, generating the correct
flags for running them, and runs them, capturing their outputs. First check is
to check the equality of the converted code output with the output of the
standard register emulator. This check would not suffice if both emulators were
incorrect, but in practice checking `by eye' was satisfactory to cover that
eventuality. Then the register code is converted with each level of optimisation
and the output of which is captured and parsed. The emulator has been written in
such a way that at a certain debug level it includes every instruction that has
been run, or information about instruction caching. After some pattern matching
on this output, numerical results are obtained.
