\chapter{Introduction}\label{ch:introduction}
\section{Aims}
This project aims to show that by converting code written for a simple
register-based architecture or processor to a stack-based architecture,
performance improvements can be found. This will include an example
implementation of translating code intended for a register-based architecture to
run on a stack-based architecture.

\section{Limitations}
Sourcing physical processors and the toolchains capable of running on both the
register-based and stack-based architectures is difficult, as what exists
already tends to be either years out of date, with only partial support, or does
not exist on both platforms. As such, this project will seek to emulate the
architectures instead.

Since the architectures will be emulated, there are precious few advantages to
continuing to use the architecture's native binary code, so this project will
instead take in some form of the respective architecture's assembly language.

\section{Statement of Ethics}
This project does not have any ethical concerns. As Transmeta have already done
a more advanced version of this project nearly two decades ago with their Crusoe
processor, any implications of the results of this project will be unlikely to
affect any commercial businesses. The software produced by this project will be
released open-source once it is completed under a permissive licence, should
anyone else want to take the project further.

\section{Problem approach}
Initially, relevant literature will be reviewed, focusing on the work done by
Transmeta, but also covering computer architectures, code translation and stack
optimisation (Chapter~\ref{ch:litreview}). The knowledge from this will be used
to make a list of requirements for the implentation of the translation program
(Chapter~\ref{ch:problemanalysis}). The program will then be designed and
implemented, documenting the problems and solutions found along the way
(Chapter~\ref{ch:designimplementation}). Once the implementation is complete, it
will be tested appropriately and results will be recorded and evalulated
(Chapter~\ref{ch:testingresults}). Finally, conclusions will be drawn and
further work discussed (Chapter~\ref{ch:conclusions}).

