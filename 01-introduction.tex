\chapter{Introduction}\label{ch:introduction}
\section{Background}
Transmeta were a US based company, who in 2000 surprised the computer
architecture industry by developing a chip which dynamically translates Intel
binary code into machine language instructions for another (highly optimised
RISC) CPU core. In doing so it allows Intel code to execute in real-time without
recompilation, but using a (claimed) much more power-efficient processor
architecture.

\section{Aims}
The objective of this project is to show that by converting code written for a
simple register-based architecture or processor to a stack-based architecture,
performance improvements can be found. This includes an implementation that can
translate code intended for a register-based architecture to run on a
stack-based architecture. Testing the output of this implementation is intended
to gain some performance related results and enable conclusions to be drawn from
them.

\section{Limitations}
Sourcing physical processors and the toolchains capable of compiling and running
code on both register-based and stack-based architectures is difficult, as
existing systems tend to be either very out of date, with only partial support,
or not exist on both platforms. Because of this, the project seeks to emulate
the architectures instead.

Since the architectures will be emulated, there are very few advantages to
continuing to use the architecture's native binary code, so this project will
instead use the respective architecture's assembly language, rather than the
binary code.

\section{Statement of Ethics}
This project has very few ethical concerns. This project only intends to
implement already published work and show that their results are correct. In
addition, Transmeta published results for their similar project over a decade
ago with their Crusoe processor and have since closed down any microprocessor
production operations and bought out, therefore any implications of the results
of this project will be unlikely to have any further affect on any commercial
businesses. There is the case of transferring the ideas from this project to
different architectures. The source architectures in this project are fully
evaluated for side effects, but changing to a different architecture may result
in different results if not examined fully, depending on the architecture. This
could cause issues for high-integrity software where correctness is crucial. The
software produced by this project has been released under a permissive
open-source licence now that it is completed, should anyone else want to take
the project any further.

\section{Problem approach}
Initially, relevant literature is reviewed, focusing on the work done by
Transmeta, but also covering computer architectures, code translation and stack
optimisation (Chapter~\ref{ch:litreview}). The knowledge from this is used to
make a list of requirements for the implentation of the translation program
(Chapter~\ref{ch:problemanalysis}). The program is then designed and
implemented, documenting the problems and solutions found along the way
(Chapter~\ref{ch:designimplementation}). The completed implementation is then
tested appropriately and results recorded and evaluated
(Chapter~\ref{ch:testingresults}). Finally, conclusions are drawn and options
for any further work set out and discussed (Chapter~\ref{ch:conclusions}).
