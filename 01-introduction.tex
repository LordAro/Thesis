\chapter{Introduction}\label{ch:introduction}
\section{Aims}
This project aims to show that by converting code written for a simple
register-based architecture or processor to a stack-based architecture,
performance improvements can be found. This will include an implementation of
translating code intended for a register-based architecture to run on a
stack-based architecture.

\section{Limitations}
Sourcing physical processors and the toolchains capable of compiling and running
code on both register-based and stack-based architectures is difficult, as
existing systems tend to be either very out of date, with only partial support,
or not exist on both platforms. As such, this project will seek to emulate the
architectures instead.

Since the architectures will be emulated, there are precious few advantages to
continuing to use the architecture's native binary code, so this project will
instead take in some form of the respective architecture's assembly language.

\section{Statement of Ethics}
This project has very few ethical concerns. This project only intends to
implement already published work and show that their results are correct. In
addition, Transmeta published results for their similar project nearly two
decades ago with their Crusoe processor and have since closed down any
microprocessor production operations and bought out, therefore any implications
of the results of this project will be unlikely to have any further affect on
any commercial businesses.  There is the case of transferring the ideas from
this project to different architectures. The source architectures in this
project will be fully evaluated for side effects, but changing to a different
architecture might result in different results if not examined fully, depending
on the architecture. This would cause issues for high-integrity software where
correctness is crucial. The software produced by this project will be released
under a permissive open-source licence once it is completed, should anyone else
want to take the project any further.

\section{Problem approach}
Initially, relevant literature will be reviewed, focusing on the work done by
Transmeta, but also covering computer architectures, code translation and stack
optimisation (Chapter~\ref{ch:litreview}). The knowledge from this will be used
to make a list of requirements for the implentation of the translation program
(Chapter~\ref{ch:problemanalysis}). The program will then be designed and
implemented, documenting the problems and solutions found along the way
(Chapter~\ref{ch:designimplementation}). Once the implementation is complete, it
will be tested appropriately and results will be recorded and evalulated
(Chapter~\ref{ch:testingresults}). Finally, conclusions will be drawn and
options for any further work set out and discussed
(Chapter~\ref{ch:conclusions}).

