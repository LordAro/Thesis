\chapter{Problem Analysis}\label{ch:problemanalysis}
The objective of this project is to create a program that can run
register-based assembly instructions on an emulated stack machine and implement
optimisation techniques that show improved execution times over the basic
implementation.

The program implementation should be broken down into a number of steps.
The first part of the implementation will be to take the chosen architectures
and implement emulators for them.  Some conversion routines for register to
stack architectures will be then implemented. After this, the generated stack
code will have optimisation techniques applied to it. Finally, suitable test
programs will be written.

\section{Requirements}
The following is a collected list of key features for the implementation of
the project that are required for the project to properly succeed:

\begin{enumerate}[noitemsep,label=R\arabic*]
  \item Fully functioning emulator for a register machine
    architecture\label{itm:register}
  \begin{enumerate}[noitemsep,label=\theenumi.\arabic*]
    \item Working parser
    \item Implement basic interpreter
    \item Implement all instructions
  \end{enumerate}
  \item Fully functioning emulator for a stack machine
    architecture\label{itm:stack}
  \begin{enumerate}[noitemsep,label=\theenumi.\arabic*]
    \item Working parser
    \item Implement basic interpreter
    \item Implement all instructions
  \end{enumerate}
  \item Ability to generate stack code from register code\label{itm:generate}
  \begin{enumerate}[noitemsep,label=\theenumi.\arabic*]
    \item Implement conversions for all register instructions
    \item Transpiler --- translate entire program and output the result
    \item Translate code in blocks at a time
  \end{enumerate}
  \item Apply optimisations to the generated stack code\label{itm:optimise}
  \begin{enumerate}[noitemsep,label=\theenumi.\arabic*]
    \item Peephole optimisations~\cite{McKeeman1965Peephole}
    \item Koopman-style stack scheduling
      optimisations~\cite{Koopman1995Preliminary}
  \end{enumerate}
  \item Series of results that show significance of optimisation of generated
    stack code\label{itm:results}
  \begin{enumerate}[noitemsep,label=\theenumi.\arabic*]
    \item Compare number of register instructions with the number of generated
      stack instructions
    \item Show reduction in instruction count of optimised stack code
    \item Show reduction in memory accesses of generated stack code after
      optimisations
  \end{enumerate}
\end{enumerate}

\noindent There are also a certain number of desirable features that are
optional for the project.

\begin{enumerate}[noitemsep,resume,label=R\arabic*]
  \item Additional stack optimisations on the generated stack
    code\label{itm:moreoptimise}
  \begin{enumerate}[noitemsep,label=\theenumi.\arabic*]
    \item Bailey's optimisation algorithm~\cite{Bailey2000Inter}
    \item Shannon's optisation algorithm~\cite{Shannon2006AC}
  \end{enumerate}
  \item Implement translation block
    caching~\cite{TransmetaCodeMorph}\label{itm:caching}
  \begin{enumerate}[noitemsep,label=\theenumi.\arabic*]
    \item Cache converted blocks
    \item Implement a translation threshold --- like the Transmeta
      implementation, only translate regions of code if they are executed more
      than a certain number of times
  \end{enumerate}
  \item Ability to more easily visualise differences between handwritten and
    generated code, e.g.\ graphs or running side-by-side\label{itm:visualise}
\end{enumerate}
%  \begin{enumerate}[noitemsep,label=\theenumi.\arabic*]
%  \end{enumerate}

The degree to which these objectives are successful will be discussed in
Section~\ref{sec:testingevaluation} and further work according to which points
are not completed will be discussed in Section~\ref{sec:furtherwork}.


\section{Architectural choices}
Prior to any programming, the programming language used has to be decided on, as
do the source (register) and target (stack) architectures.

An early decision made was to write emulators for the register and stack
machines chosen. Toolchains for `obscure' architectures such as the ones likely
to be chosen tend to be rather limited in nature and either out-of-date, broken
or both. Writing emulators for the source and target architectures creates extra
programming work, but means that the choices of programming language and style
are much greater.

It was decided that to simplify the design, the program would be to
interpret assembly source code of the register and stack architectures instead
of running the compiled binary. Doing it this way is only an abstraction over
actually reading the binaries, and skips out having to compile the source code
and then decode it again. It would not be difficult to make a program that does
this but was deemed unnecessary for the core part of this project, to convert a
register-based instruction set to a stack-based one.

\subsection{Programming language}
The advantages and disadvantages of several programming languages were evaluated
for the project.

\subsubsection{C}
C is one of the oldest programming languages in the world and is only just a
step above the `raw' assembly level. Despite the age of the language, its
low-level nature makes it extremely fast, often being the fastest of all
languages, as it has over 40 years of compiler theory and optimisations backing
it up. With the emulators written in C there would almost no concern for their
speed.  However, it does mean that the programmer has to do everything
themselves, like handle memory management, and it has very little in the way of
type system which makes it easy to make mistakes in and slows down development
of features.

\subsubsection{C++}
C++ is a systems programming language that is well known for its hugely
powerful templates for generic programming, its ability to use multiparadigm
styles of programming and its high speed, with significant work being put into
optimising compilers for the language. Using object-orientated programming, the
conversion routines can be built on top of one of the emulators, with little
duplication of code.

The language's power also comes with some disadvantage however. It makes it easy
to make errors in programming and logic, and its compiler errors for templates
are infamous for taking up many pages. In fact, C++'s own creator, Bjarne
Stroustrup, has been quoted as saying ``C makes it easy to shoot yourself in the
foot; C++ makes it harder, but when you do it blows your whole leg off'' when
comparing C++ to its parent language C. That said, modern C++ has been making
big improvements to the language since 2011, with smart pointers, lambdas \&
type deduction fixing many of the language's major criticisms and making it
almost unnecessary to ever use manage the program's memory manually.

\subsubsection{Rust}
Rust is a relatively new systems language that takes inspirations from C++ and
Haskell, with its main selling point being guaranteed memory safety and
`zero-cost abstractions' which makes the language very fast to use and very
difficult to break in the standard ways with segmentation faults and similar
errors. This safety comes at a cost however as it has a very steep learning
curve and the compiler's `borrow checker' (how it tracks where values in memory
are used) is already gaining a reputation for being hard deal with and make
programs correct to the point that they will compile.

\subsubsection{Python}
Python is a higher-level language that has a syntax that allows concepts to be
expressed quickly and easily, with resource management being handled by the
language. Unlike the other languages discussed here, its type system is dynamic,
meaning that variables can be reassigned to different types, which helps with
the ease of programming in it. Also unlike the other languages here
it is an interpreted language (instead of compiled). While this has its
advantages, mostly with the speed of development, it has a big impact on the
speed of the language by comparison.

\subsubsection{Go}
Go (often called Golang) is also a new systems programming language. It's
strongly inspired by C, but adds several features that make it easier to use.
The main feature is that the language is garbage collected, meaning no manual
memory management. Go's garbage collector is highly sophisticated too, boasting
often getting microsecond times for its `Stop the World' phase. The language
also includes several concurrency features that make it a lot easier to program
parallel code, in particular for web services. Go's garbage collector is also a
common criticism as systems programming often require manual memory management
due to system restraints, which the language offers no ability to do so. Go's
designers have also taken the decision to leave certain features out of the
language such as generics and operator overloading, which leads to duplicated
code.

Therefore it was decided to use C++ as this would be a reasonable compromise
between its speed and multi-paradigm programming ability, over the disadvantages
of the risk of reduced development speed from fixing programming errors and the
language's verbosity over a high-level language.

\subsection{Architecture choices}

\subsubsection{Register architectures}
A shortlist of register architectures was drawn up, of the Zilog Z80, the Xilinx
Picoblaze, and the DCPU-16. The Z80 is a microprocessor introduced in 1976. It
is 8-bit based with the capability to address 64KB of memory by means of
combining its registers. An extremely popular processor during the 1980s, it is
still used in embedded systems. It has its own assembly
language that has both register and stack elements. The Picoblaze is a `soft
processor core', meaning that isn't manufactured specially, rather is fabricated
on an FPGA (field-programmable gate array) by the user. It uses an 8-bit RISC
architecture, which makes it very simple to fabricate and run. The DCPU-16 isn't
a real processor, and has never been produced. It is the invention of
Markus Persson in 2011 to be emulated as part of a game he was creating that
never completed, but there was significant interest in it, which means that
there was many programs and emulators produced for the DCPU-16 processor.
It's a 16-bit processor that has modularity in mind in the architecture --- the
computer would `connect' to peripherals to do its IO instead of a more common
console output interface. Listings~\ref{lst:z80asm},\ref{lst:picoasm}
and~\ref{lst:dcpuasm} show examples of a ``Hello World'' program in the
respective assembly languages of the Z80, Picoblaze and DCPU-16.

\noindent\begin{minipage}{0.5\textwidth}
\begin{lstlisting}[caption={Z80 ASM},label={lst:z80asm}]
      LD a, 0
      LD (CURCOL), a
      LD (CURROW), a
      LD hl, text
      B_CALL(_PutS)
      RET
text:
      .db "Hello, world!", 0
\end{lstlisting}%
\end{minipage}%
\noindent\begin{minipage}{0.5\textwidth}
\begin{lstlisting}[caption={Picoblaze ASM},label={lst:picoasm}]
module hello_world;

initial begin
  $display ("Hello, world!");
  #10 $finish;
end

endmodule
\end{lstlisting}%
\end{minipage}

\noindent\begin{minipage}{\linewidth} %prevent page break
\begin{lstlisting}[caption={DCPU-16 ASM}, label={lst:dcpuasm}]
; Attach screen
SET A, 0
SET B, vram
HWI 0

SET J, 0

:loop
SET I, vram
ADD I, J
ADD [I], 0x2000
ADD J, 1
IFN J, 12
    SET PC, loop

:crash
SET PC, crash

:vram
DAT "Hello, world!", 0
\end{lstlisting}%
\end{minipage}

As the Picoblaze is a soft-processor, it is separated into its ASM and Verilog
components and you have to processor's components have to be connected together
manually as part of the build process. The Picoblaze was there fore discounted
as it was decided that this would be too much boilerplate.
After some initial emulator implementation for the
Z80 processor, it was found that its ability to combine registers was not easy
to implement in a good way in C++, so after some thought it was decided to
continue with the DCPU-16 assembly language (specification included as
Appendix~\ref{ch:dcpuspec}), after cutting it down to get rid of
the peripherals part of the language which isn't necessary for this project.

\subsubsection{Stack architectures}

In terms of stack architectures, Forth, Jasmin, and J5 were shortlisted. Forth
is a very early programming languages that dates from 1970, when stack machines
were still common in everyday computing. It's a very simple compiled language
that still finds uses today in embedded systems due to how easy it is to
implement in new systems, and its low memory overhead. Jasmin is a variant of
JVM bytecode, meaning it is a text format rather than the binary files of raw
JVM bytecode. J5 is a teaching stack language to introduce people to stack
machines and their respective languages and it is very simple in that regard,
drawing on ideas from various different actual stack languages. Examples of
`Hello World' programs in Forth, J5 and Jasmin can be seen in
Listings~\ref{lst:forthasm},\ref{lst:j5asm} and~\ref{lst:jasminasm}
respectively.

\noindent\begin{minipage}{0.5\textwidth}
\begin{lstlisting}[caption={Forth ASM},label={lst:forthasm}]
CR .( Hello, world!)
\end{lstlisting}
\end{minipage}%
\noindent\begin{minipage}{0.5\textwidth}
\begin{lstlisting}[caption={J5 ASM},label={lst:j5asm}]
OUT "Hello, world!"
\end{lstlisting}
\end{minipage}
\begin{minipage}{\textwidth}
\begin{lstlisting}[caption={Jasmin ASM}, label={lst:jasminasm}]
.class public HelloWorld
.super java/lang/Object

.method public static main([Ljava/lang/String;)V
  .limit stack 3
  .limit locals 1
  getstatic      java/lang/System/out Ljava/io/PrintStream;
  ldc            "Hello World."
  invokevirtual  java/io/PrintStream/println(Ljava/lang/String;)V
  return
.end method
\end{lstlisting}
\end{minipage}

Given Jasmin's complexity to write and Forth's complexity to implement, it was
decided to target the J5 stack language (specification included as
Appendix~\ref{ch:j5spec}) due to its simplicity compared to the other two
languages.
