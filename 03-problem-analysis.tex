\chapter{Problem Analysis}\label{ch:problemanalysis}
The objective of this project is to create a program that can run
register-based assembly instructions on an emulated stack machine, and implement
optimisation techniques that show improved execution times over the basic
implementation.

The program implementation should be broken down into a number of steps. First
will be deciding on the source (register) and target (stack) architectures.  The
first part of the implementation will be to implement emulators for the chosen
architectures.  Some conversion routines for register to stack architectures
will be then implemented. After this, the generated stack code will have
optimisation techniques applied to it. Finally, suitable test programs will be
written.

\section{Requirements}
The following is a codified list of key features for the implementation of
the project that are required for the project to succeed:

\begin{itemize}[noitemsep]
  \item Fully functioning emulator for a register machine architecture.
  \item Fully functioning emulator for a stack machine architecture.
  \item Ability to generate stack code from register code.
  \item Peephole optimisations on generated stack code.
  \item Koopman-style stack scheduling optimisations on generated stack code.
  \item Series of results that show significance of optimisation of generated
    stack code.
\end{itemize}

There are also a certain number of desirable features that are optional for the
project.

\begin{itemize}[noitemsep]
  \item Bailey \& Shannon optimisations on generated stack code.
  \item Translation threshold --- like the TransMeta implementation, only
  translate regions of code if they are executed more than a certain number of
  times.
  \item Ability to more easily visualise differences between handwritten and
  generated code, e.g.\ graphs or running side-by-side.
\end{itemize}

The degree to which these objectives are successful will be discussed in
Section~\ref{sec:testingevaluation} and further work according to which points
are not copmleted will be discussed in Section~\ref{sec:furtherwork}.
